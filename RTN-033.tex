\documentclass[authoryear,toc]{lsstdoc}
% lsstdoc documentation: https://lsst-texmf.lsst.io/lsstdoc.html
\input{meta}

% Package imports go here.

% Local commands go here.

%If you want glossaries
%\input{aglossary.tex}
%\makeglossaries

\title{The In-Kind Helpdesk System}

% Optional subtitle
% \setDocSubtitle{A subtitle}

\author{%
Steve Margheim, 
Aprajita Verma,
Phil Marshall
}

\setDocRef{RTN-033}
\setDocUpstreamLocation{\url{https://github.com/lsst/rtn-033}}

\date{\vcsDate}

% Optional: name of the document's curator
% \setDocCurator{The Curator of this Document}

\setDocAbstract{%
We use Jira and the Jarvis auto-ticketer to implement an outward-facing Helpdesk system, with configuration and operating procedure designed for use by the Rubin LSST in-kind program contribution teams and recipients as they seek assistance from the In-kind Program Coordinators (IPC) in the Rubin operations IPC Team. The system follows an initial design by the Rubin construction project Communications  (COMT) and IT teams. This technote describes the system as its use.
}

% Change history defined here.
% Order: oldest first.
% Fields: VERSION, DATE, DESCRIPTION, OWNER NAME.
% See LPM-51 for version number policy.
\setDocChangeRecord{%
  \addtohist{0.1}{2022-03-18}{Initial draft.}{Steve Margheim}
}

%%%%%%%%%%%%%%%%%%%%%%%%%%%%%%%%%%%%%%%%%%%%%%%%%%%%%%%%%%%%%%%%%%%%%%%%
%%%%%%%%%%%%%%%%%%%%%%%%%%%%%%%%%%%%%%%%%%%%%%%%%%%%%%%%%%%%%%%%%%%%%%%%

\begin{document}

% Create the title page.
\maketitle
% Frequently for a technote we do not want a title page  uncomment this to remove the title page and changelog.
% use \mkshorttitle to remove the extra pages

% ADD CONTENT HERE
% You can also use the \input command to include several content files.

%%%%%%%%%%%%%%%%%%%%%%%%%%%%%%%%%%%%%%%%%%%%%%%%%%%%%%%%%%%%%%%%%%%%%%%%

\section{Requirements} \label{sec:reqs}

We first outline some use cases for a helpdesk system to support the ``users'' of the in-kind program, namely the contribution teams, program managers, and recipients -- as well as the In-kind Program Coordinator (IPC) respondents. 
From these examples we derive a simple set of requirements for the system.

%%%%%%%%%%%%%%%%%%%%%%%%%%%%%%%%%%%%%%%%%%%%%%%%%%%%%%%%%%%%%%%%%%%%%%%%

\section{Design} \label{sec:design}

We now describe the design of the In-Kind Helpdesk system. 
The helpdesk is implemented as a Jira project, with a dedicated email list connected to it.


%%%%%%%%%%%%%%%%%%%%%%%%%%%%%%%%%%%%%%%%%%%%%%%%%%%%%%%%%%%%%%%%%%%%%%%%

\section{Operation} \label{sec:operation}

This section provides a set of how-to's, for reference by the IPCs on helpdesk duty.

%%%%%%%%%%%%%%%%%%%%%%%%%%%%%%%%%%%%%%%%%%%%%%%%%%%%%%%%%%%%%%%%%%%%%%%%

\section{Future Work} \label{sec:future}
 
The In-kind Helpdesk is now in use by the IPCs and in-kind users. 
In future, we could think of making the following improvements:

\begin{itemize}
\item \ldots
\end{itemize}
 
%%%%%%%%%%%%%%%%%%%%%%%%%%%%%%%%%%%%%%%%%%%%%%%%%%%%%%%%%%%%%%%%%%%%%%%%

\appendix
% Include all the relevant bib files.
% https://lsst-texmf.lsst.io/lsstdoc.html#bibliographies
\section{References} \label{sec:bib}
\renewcommand{\refname}{} % Suppress default Bibliography section
\bibliography{local,lsst,lsst-dm,refs_ads,refs,books}

% Make sure lsst-texmf/bin/generateAcronyms.py is in your path
\section{Acronyms} \label{sec:acronyms}
\input{acronyms.tex}
% If you want glossary uncomment below -- comment out the two lines above
%\printglossaries

\end{document}
